% Notes Template
% Template version: v0.4.1
% https://github.com/rnsavinelli/notes-template
%
%!TEX encoding = UTF-8 Unicode

% Document-class options:
\documentclass[letter]{article}

% Package includes and formatting settings -----------------------------------
% ----------------------------------------------------------------------------

% LaTeX Font encoding - 8-bit fonts
\usepackage[T1]{fontenc}
\usepackage[utf8]{inputenc}
\usepackage[sc, osf]{mathpazo}
\usepackage[spanish]{babel}

\usepackage{hyperref}

% These settings should be modified by the user
\def\authorsname{Savinelli Roberto Nicolás, Parente Franco Daniel, Fuentebuena Guardon Tobías.}
\def\authorswebsite{}
\def\course{Sintaxis y Semántica de los Lenguajes K2053}
\def\courseyear{2021}
\def\thetitle{Lenguaje de programación MINI}
\def\thesubtitle{Segundo trabajo práctico}
\def\university{Universidad Tecnológica Nacional}

% PDF Settings
\hypersetup{
    pdfauthor = {\authorsname},
    pdfkeywords = {\course, \university},
    pdftitle = {\thetitle, \thesubtitle},
    pdfsubject = {-},
}

% HREF settings
\hypersetup{
    colorlinks = true,
    urlcolor = blue,
    linkcolor = black
}

% LaTeX' own graphics handling
\usepackage{graphicx}
\graphicspath{ {./images/} }

% AMS-LaTeX extensions for mathematical typesetting.
\usepackage{amsmath,amsthm,amsfonts,amssymb,mathrsfs}

% Document format and settings -----------------------------------------------
% ----------------------------------------------------------------------------
\usepackage[left=1in, right=1in, bottom=1.25in, top=1.25in]{geometry}

% Better spacing between lines
\usepackage{parskip}

\setlength\parindent{0em}
%\pagenumbering{roman}

\usepackage{chngcntr}
\counterwithin*{equation}{section}

% Title, section, and topic formatting
\usepackage{titlesec}
\usepackage[labelfont=bf]{caption}

% \today commands
\usepackage{datetime}

% Content of the document ----------------------------------------------------
% ----------------------------------------------------------------------------

\begin{document}

\title{\thetitle \\ {\large \thesubtitle}}
\author{\authorsname}
\date{\today}

\maketitle

% Include as many files as needed. This approach is taken to reduce the amount
% of information contained inside this very same document. It is not mandatory
% to use external files to add your content, but it is advisable.
%
% \input is preferred over \include since the latter automatically appends a
% page break. Deppending of your use case you might prefer one over the other.

%!TEX root = notex.tex

\section{Introducción}

Con el desarrollo del presente trabajo práctico se busca formalizar el lenguaje de programación MINI diseñado por Lic. Eduardo Pablo Zúñiga para el curso \course \ de \  \university \ (UTN) Facultad Regional Buenos Aires, como se dictó en 2021.


%!TEX root = notex.tex

\section{Gramática Léxica}

A continuacion describimos las producciones de la gramática léxica del lenguaje de programación MINI.

\subsection{Categorías Léxicas}

\begin{tabbing}

\textit{token}: \= \+ \\
    \textit{identificador} \\
    \textit{constante}\\
    \textit{palabra-reservada} \\
    \textit{operador} \\
    \textit{asignación} \\
    \textit{caracter-de-puntuación}

\end{tabbing}

\subsection{Identificadores}

\begin{tabbing}

\textit{identificador}: \= \+ \\
    \textit{letra} \\
    \textit{identificador} \textit{letra} \\
    \textit{identificador} \textit{digito} \\

\- \\
\textit{letra}: uno de \+ \\
    \textbf{a b c d e f g h i j k l m n o p q r s t u v w x y z}\\
    \textbf{A B C D E F G H I J K L M N O P Q R S T U V W X Y Z} \\

\- \\
\textit{digito}: uno de \+ \\
    \textbf{0 1 2 3 4 5 6 7 8 9}

\end{tabbing}

\subsection{Constantes}

\begin{tabbing}

\textit{constante}: \= \+ \\
    \textit{digito} \\
    \textit{constante} \textit{digito}\\

\- \\
\textit{digito}: uno de \+ \\
    \textbf{0 1 2 3 4 5 6 7 8 9}

\end{tabbing}

\subsection{Palabras Reservadas}

\begin{tabbing}

\textit{palabra-reservada}: \= uno de \+ \\
    \textbf{programa entero leer escribir fin-programa}

\end{tabbing}

\subsection{Operadores}

\begin{tabbing}

\textit{operador}: \= uno de \+ \\
    \textbf{
        +
        \hspace{4pt}
        -
        \hspace{4pt}
        *
        \hspace{4pt}
        /
        \hspace{4pt}
        \%
    }

\end{tabbing}

\subsection{Asignación}

\begin{tabbing}

\textit{asignación}: \= \+ \\
    \textbf{<}\textbf{<}

\end{tabbing}

\subsection{Caracteres de Puntuación}

\begin{tabbing}

\textit{caracter-de-puntuación}: \= uno de \+ \\
    \textbf{
        (
        \hspace{4pt}
        )
        \hspace{4pt}
        ;
    }

\end{tabbing}

%!TEX root = notex.tex

\section{Gramática Sintáctica}

A continuación se describen las producciones de su gramática sintáctica.

\subsection{Programa}

\begin{tabbing}

\textit{programa}: \= \+ \\
    \textbf{programa} \textit{identificador}
    \textit{logica}
    \textbf{fin-programa} \\

\- \\ \textit{logica}: \+ \\
    \textit{sentencia} \\
    \textit{logica} \textit{sentencia} \\

\- \\ \textit{sentencia}: \+ \\
    \textit{identificador} \textbf{<}\textbf{<} \textit{expresión} \textbf{;}\\
    \textbf{entero} \textit{identificador} \textbf{;}\\
    \textbf{leer} \textbf{(} \textit{lista-de-identificadores} \textbf{)} \textbf{;}\\
    \textbf{escribir} \textbf{(} \textit{lista-de-expresiones} \textbf{)} \textbf{;}\\

\- \\ \textit{lista-de-identificadores}: \+ \\
    \textit{identificador} \\
    \textit{lista-de-identificadores} \textbf{,} \textit{identificador} \\

\- \\ \textit{lista-de-expresiones}: \+ \\
    \textit{expresión} \\
    \textit{lista-de-expresiones} \textbf{,} \textit{expresión} \\

\- \\ \textit{expresión}: \+ \\
    \textit{expresión-secundaria} \\
    \textit{expresión-secundaria} \textit{operador-primario} \textit{expresión-secundaria}\\

\- \\ \textit{expresión-secundaria}: \+ \\
    \textit{expresión-primaria} \\
    \textit{expresión-primaria} \textit{operador-secundario} \textit{expresión-primaria}\\

\- \\ \textit{expresión-primaria}: \+ \\
    \textit{identificador} \\
    \textit{constante} \\
    \textbf{(} \textit{expresión} \textbf{)}\\

\end{tabbing}



% Closing of the document ----------------------------------------------------
% ----------------------------------------------------------------------------
% Comment the commands below to re-enable section,
% subsection and subsubsection numbering.

\titleformat{\section}{\bfseries\Large}{}{0em}{}
\titleformat{\subsection}{\bfseries\large}{}{0em}{}
\titleformat{\subsubsection}{\bfseries\normalsize}{}{0em}{}

% Bibliography ---------------------------------------------------------------
% ----------------------------------------------------------------------------

\newpage
\bibliographystyle{plain}
\bibliography{bib/ssl}

% Appendices -----------------------------------------------------------------
% ----------------------------------------------------------------------------

%\newpage
%\input{appendix-a}

% About Section --------------------------------------------------------------
% ----------------------------------------------------------------------------

%\newpage
%%!TEX root = notex.tex

%\section{About this document}

%Last updated: {\today}.


\end{document}

% End of the document --------------------------------------------------------
% ----------------------------------------------------------------------------
