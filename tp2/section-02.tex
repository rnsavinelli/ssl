%!TEX root = notex.tex

\section{Gramática Sintáctica}

A continuación se describen las producciones de su gramática sintáctica.

\subsection{Programa}

\begin{tabbing}

\textit{programa}: \= \+ \\
    \textbf{programa} \textit{identificador}
    \textit{logica}
    \textbf{fin-programa} \\

\- \\ \textit{logica}: \+ \\
    \textit{sentencia} \\
    \textit{logica} \textit{sentencia} \\

\- \\ \textit{sentencia}: \+ \\
    \textit{identificador} \textbf{<}\textbf{<} \textit{expresión} \textbf{;}\\
    \textbf{entero} \textit{identificador} \textbf{;}\\
    \textbf{leer} \textbf{(} \textit{lista-de-identificadores} \textbf{)} \textbf{;}\\
    \textbf{escribir} \textbf{(} \textit{lista-de-expresiones} \textbf{)} \textbf{;}\\

\- \\ \textit{lista-de-identificadores}: \+ \\
    \textit{identificador} \\
    \textit{lista-de-identificadores} \textbf{,} \textit{identificador} \\

\- \\ \textit{lista-de-expresiones}: \+ \\
    \textit{expresión} \\
    \textit{lista-de-expresiones} \textbf{,} \textit{expresión} \\

\- \\ \textit{expresión}: \+ \\
    \textit{expresión-secundaria} \\
    \textit{expresión-secundaria} \textit{operador-primario} \textit{expresión-secundaria}\\

\- \\ \textit{expresión-secundaria}: \+ \\
    \textit{expresión-primaria} \\
    \textit{expresión-primaria} \textit{operador-secundario} \textit{expresión-primaria}\\

\- \\ \textit{expresión-primaria}: \+ \\
    \textit{identificador} \\
    \textit{constante} \\
    \textbf{(} \textit{expresión} \textbf{)}\\

\end{tabbing}

