%!TEX root = notex.tex

\section{Gramática Léxica}

En esta sección se describen las producciones de la gramática léxica del lenguaje de programación MINI.

\subsection{Categorías Léxicas}

\begin{tabbing}

\textit{token}: \= \+ \\
    \textit{identificador} \\
    \textit{constante}\\
    \textit{palabra-reservada} \\
    \textit{operador} \\
    \textit{asignación} \\
    \textit{caracter-de-puntuación}

\end{tabbing}

\subsection{Identificadores}

\begin{tabbing}

\textit{identificador}: \= \+ \\
    \textit{letra} \\
    \textit{identificador} \textit{letra} \\
    \textit{identificador} \textit{digito} \\

\- \\
\textit{letra}: una de \+ \\
    \textbf{a b c d e f g h i j k l m n o p q r s t u v w x y z}\\
    \textbf{A B C D E F G H I J K L M N O P Q R S T U V W X Y Z} \\

\- \\
\textit{digito}: uno de \+ \\
    \textbf{0 1 2 3 4 5 6 7 8 9}

\end{tabbing}

\subsection{Constantes}

\begin{tabbing}

\textit{constante}: \= \+ \\
    \textit{constante-negativa} \\
    \textit{constante-positiva} \\

\- \\
\textit{constante-negativa}: \+ \\
    \textbf{-} \textit{constante-positiva} \\

\- \\
\textit{constante-positiva}: \+ \\
    \textit{digito} \\
    \textit{constante-positiva} \textit{digito}\\

\- \\
\textit{digito}: uno de \+ \\
    \textbf{0 1 2 3 4 5 6 7 8 9}

\end{tabbing}

\subsection{Palabras Reservadas}

\begin{tabbing}

\textit{palabra-reservada}: \= \+ \\
    \textbf{programa}\\
    \textbf{entero}\\
    \textbf{leer}\\
    \textbf{escribir}\\
    \textbf{fin-programa}

\end{tabbing}

\subsection{Operadores}

\begin{tabbing}

\textit{operador}: \= \+ \\
    \textit{operador-primario} \\
    \textit{operador-secundario} \\

\- \\ \textit{operador-primario}: uno de \+ \\
    \textbf{
        +
        \hspace{3pt}
        -
    }\\

\- \\ \textit{operador-secundario}: uno de \+ \\
    \textbf{
        *
        \hspace{3pt}
        /
        \hspace{3pt}
        \%
    }

\end{tabbing}

\subsection{Asignación}

\begin{tabbing}

\textit{asignación}: \= \+ \\
    \textbf{<}\textbf{<}

\end{tabbing}

\subsection{Caracteres de Puntuación}

\begin{tabbing}

\textit{caracter-de-puntuación}: \= uno de \+ \\
    \textbf{
        (
        \hspace{3pt}
        )
        \hspace{3pt}
        ;
        \hspace{3pt}
        ,
    }

\end{tabbing}

\subsection{Caracteres de Puntuación}

Los comentarios son ignorados por el scanner. Los mismos comienzan con // y finalizan con el caracter de nueva línea.
